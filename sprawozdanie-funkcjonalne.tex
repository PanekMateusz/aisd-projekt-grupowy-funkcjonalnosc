\documentclass[10pt,a4paper]{article}
\usepackage[a4paper]{geometry}

\usepackage{polski}
\usepackage{xltxtra}
\usepackage{indentfirst}
\usepackage{relsize}
\usepackage{fancyvrb}
\usepackage{hyperref}
\hypersetup{
    pdftitle={},%
    pdfauthor={},%
    colorlinks=true,        % false: boxed links; true: colored links
    linkcolor=black,        % color of internal links
    citecolor=green,        % color of links to bibliography
    filecolor=magenta,      % color of file links
    urlcolor=cyan,          % color of external links
    unicode=true,           % non-Latin characters in Acrobat’s bookmarks
    pdfstartview={FitH},    % fits the width of the page to the window
    pdfnewwindow=true       % links in new window
}

%% tweak fonts
\defaultfontfeatures{Mapping=tex-text}
\setromanfont{Charis SIL}
%%\setsansfont[Scale=MatchLowercase]{Helvetica Neue}
%%\setmonofont[Scale=MatchLowercase]{Menlo}
%%\linespread{1.25}

%% define custom commands and environments
\DefineVerbatimEnvironment%
  {SmallVerbatim}%
  {Verbatim}{fontsize=\relsize{-0.5},numbers=left,numbersep=-10pt,frame=lines,tabsize=4}

\newcommand{\f}[1]{\texttt{#1}}
\newcommand{\s}[1]{\textsf{#1}}

\begin{document}

\title{Specyfikacja funckjonalna\\
  Projekt grupowy z przedmiotu AiSD
}
\author{
  Tomasz Cudziło\\
  Mateusz Malicki\\
  Mateusz Ochtera\\
  \textsc{PW EE Informatyka}\\[10pt]
}
\date{\today}
\maketitle



\section{Opis projektu}

Zadanie polega na zaimplementowaniu algorytmu genetycznego służącego do
optymalizacji wnętrza środka komunikacji miejskiej oraz wizualizację wyników.


\subsection{Wejście}
Użytkownik podaje:
\begin{itemize}
  \item maksymalny koszt produkcji autobusu,
  \item ustalona szerokość autobusu,
  \item maksymalny długość autobusu,
  \item minimalną ilość miejsc w autobusie (dowolnie siedzących lub stojących),
  \item minimalna ilość wejść (wejścia są rozmieszczane automatycznie i
    równomiernie).
\end{itemize}
Opcjonalnie, możliwe będzie również podanie maksymalnej wielkości populacji oraz
maksymalną ilość generacji.


\subsection{Wyjście}
Bitmapa z wizualizacją wnętrza autobusu, oraz wypisanymi statystykami tj.:
liczba miejsc siedzących, liczba miejsc stojących, szerokość i długość autobusu
oraz koszt produkcji.



\section{Opis działania programu}


\subsection{Funkcja przystosowania}
Brane pod uwagę są następujące czynniki:
\begin{itemize}
    \item pojemność autobusu --- maksymalna, niezależna od komfortu pasażerów,
    \item odległość wyjść od miejsc,
    \item koszt produkcji autobusu,
    \item zagęszczenie miejsc stojących --- pasażerowie, o ile to możliwe,
      preferują być oddaleni od współpasażerów o minimum 50 cm.
\end{itemize}


\subsection{Selekcja}
\emph{Metoda turniejowa} --- populację dzielimy na szereg dowolnie licznych
grup. Następnie, z każdej z nich, wybrany zostanie osobnik o najlepszym
przystosowaniu.


\subsection{Reprezentacja} 
Osobniki oraz ich genotyp prezentowane bedą w postaci obiektu:\\

\begin{SmallVerbatim}
    class bus{
	    private double sizeX;
	    private double sizeY;
	    private int seatNb;
	    private double[][] seatAllign;
    }
\end{SmallVerbatim}

\begin{itemize}
	\item sizeX - długość autobusu;
	\item sizeY - szerokość autobusu;
	\item seatNb - liczba miejsc siedzących
	\item seatAllign - współrzędne kolejnych siedzeń w autobusie
\end{itemize}
	

\subsection{Krzyżowanie}
Wymiary potomstwa będą średnią wymiarów rodziców, podobnie ilość miejsc.
Następnie losujemy kolejne miejsca przy czym te które się powtórzyły u obydwojga
rodziców mają siłą rzeczy dwa razy większą szanse na znalezienie się u
potomstwa.

\subsection{Współczynnik mutacji}
Moim zdaniem chodziło mu o opisanie rodzaju mutacji. Czyli np. będziemy
wprowadzać niewielkie odchylenia w losowych osobnikach. Zczytać aktualny
rozmiar i ilości miejsc i odpowiednio troszeczkę je zmodyfikować w jedną bądź
drugą stroną.  A współczynnik powinien być stosunkowo niewielki żeby nie
stworzyć armii mutantów u których pozanikają cechy rodziców. Mateusz O. 


\end{document}
